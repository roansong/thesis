\chapter{Literature Review}
\section{Classification}

\subsection{Nearest-Neighbour Classification}
\subsubsection{Description}
The nearest neighbour classifier operates as follows:\\
When given an input, the classifier compares this input to the training data set, and finds the one that is closest to the input. For example, if men and women were to be classified by their heights, a given input would be classified as either male or female based on the data point in the training data with the height closest to that of the input. This can be expanded to multiple features/dimensions by taking the Euclidean distance between the input and each instance in the training data set. For images, this amounts to comparing, pixel by pixel, each pixel value.\\\\
\subsubsection{K-Nearest-Neighbour Method}

\subsubsection{Defining 'Distance' between Images}
As mentioned above, the pixel values are compared directly. This can result in misleading classifications, as pixels in images are typically related in some way to the pixels near to them, and taking individual Euclidean distances fails to account for this.\\
These problems can be circumvented through the use of kernel methods, or various other preprocessing methods to take the pixel-to-pixel relationships into account. The MSTAR dataset has a collection of fixed-size images, with the target in the exact center, so it serves as a good example of pre-processed data. This simplifies the  
\paragraph{IMED}\cite{IMED}
\paragraph{Kernel Methods}
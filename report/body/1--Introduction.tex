\chapter{Introduction}

%7593 as of 21/10
%5763
As processing power has become more readily available over the years, the field of machine learning has seen many innovations. Deep learning techniques were once considered only theoretically viable, but are now being used to great effect in commercial ventures, including image classification and facial recognition applications. Their success lies in the ability of neural networks to train themselves under human supervision,  extracting features from a dataset that are not always intuitive in a human sense, and in many cases providing greater accuracy in classification than humans can achieve, in a fraction of the time (cite some medical science paper here). This accuracy and speed of classification lends itself to the field of target acquisition and recognition; recognising and classifying a target based on its radar signature, to allow for human action to be taken. Na{\"i}ve classification methods have proven to be somewhat effective, and the aim of this study is to assess the possible application of deep learning techniques in this field.

For this study, the MSTAR dataset of radar images will be used to train the various classifiers. The Nearest Neighbour classifier will be used as a benchmark against which the effectiveness of a neural network-based approach will be compared. The hope of this study is that the use of neural networks will allow for significantly higher classification accuracy, with the potential extension to different datasets without compromising the effectiveness of the classifier.

%5998

\section{Background}

\section{Motivation}

\section{Objectives}

\section{Scope and Limitations}

\section{Report Overview}


% State objective and brief overview
% Lead-in to background and motivation
% Important Terminology
% Background / Motivation
% Objective  / Hypothesis
% Problem description
% Focus
% Scope and limitations
% Document outline

\chapter{Introduction}

%7593 as of 21/10
%5763

This thesis aims to assess the effectiveness of deep learning techniques in the classification of radar imagery. Deep learning relies on the use of neural networks; interconnected layers of nodes sharing information and undergoing non-linear transformations that, through training and optimisation, can detect and extract features from a dataset without human supervision. Training a classifier on a known set of instances allows it to build a predictive model that can then be tested on unseen data, with varying degrees of accuracy. In the specific case of this report, the instances are radar images from the MSTAR dataset.

%The better the optimisation and training of the model, the greater its predictive power.




%5998

\section{Background}
The desire to mimic human brain function has driven the development of artificial intelligence (AI) and deep learning. The human brain can be viewed as a series of interconnected neurons, firing when undergoing different stimuli. This view led to the foundation of modern deep learning techniques in the 1940s, using multiple layers of artificial neurons. Due to hardware limitations, this approach saw neither success nor widespread adoption. The 21\textsuperscript{st} century has seen renewed interest in the field due to increased computational capabilities. While the complexity of the human brain is currently beyond accurate emulation, the deep learning techniques used to approximate it have found their uses in commercial classification problems.

\section{Motivation}

Deep learning is used commercially in voice and image recognition, recommendation engines, artificial intelligence, and a host of other applications. Deep learning classifiers are characterised by relatively long training times and fast classification, making them suited to real-time applications; the time taken to train a predictive model is much greater than the time taken to classify a specific instance, but it can be done beforehand on known data. Acknowledging the success of deep learning techniques has encouraged this report to test the applicability of deep learning techniques in target acquisition and classification of radar imagery.  


\section{Objectives}

This report aims to develop two classifiers suitable for use on the MSTAR dataset: 

\begin{enumerate}
	\item K-Nearest Neighbours classifier, supporting user-selected values of K
	\item Multilayer Perceptron, supporting multiple hidden layers, where each layer's size and activation function can be specified.
\end{enumerate}

Each classifier will be compared according to the following performance metrics:
\begin{itemize}
	\item Training time
	\item Training accuracy
	\item Classification time
	\item Classification accuracy
\end{itemize}

The steps to be taken in this report:
\begin{itemize}
	\item Understand the format of the MSTAR dataset
	\item Perform pre-processing of the MSTAR dataset
	\item Develop the KNN classifier
	\item Develop the Multilayer Perceptron
	\item Collect performance data for each classifier
	\item Compare and contrast each classifier based on their performance metrics
	\item Establish the merit of deep learning techniques in the context of target recognition
\end{itemize}



\section{Scope and Limitations}

\subsection{Focus}
This focus of this report is to:
\begin{itemize}
	\item Review the appropriate academic literature regarding deep learning, and assess the current body of knowledge on the subject to understand where this report will be able to contribute
	\item Implement KNN and Multilayer Perceptron classifiers in Python
	\item Implement a training and testing regime for each classifier in Python
	\item Determine and comment on which classifier is most suitable for the task of radar target recognition
\end{itemize}
\subsection{Scope}
Within project scope:
%\begin{itemize}
%	\item A literature review of the appropriate knowledge pertinent to this report
%\end{itemize}
Outside project scope:
%\begin{itemize}
%	\item 
%\end{itemize}
\subsection{Limitations}
All computation will be performed on a desktop computer running Windows 10 Pro with an Intel i5-2500 processor (four cores @ 3.3-3.7GHz), Samsung Evo 850 SSD, and 8GB of DDR3 RAM .




\section{Report Overview}


% State objective and brief overview
% Lead-in to background and motivation
% Important Terminology
% Background / Motivation
% Objective  / Hypothesis
% Problem description
% Focus
% Scope and limitations
% Document outline

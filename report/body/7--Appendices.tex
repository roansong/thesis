\appendix
\chapter{Progress Report}

As of the 16th of September, certain progress has been made in three areas of the report, namely the Literature Review, Methodology, and Implementation.


\section{Literature Review}
The Literature Review has been somewhat written; the core topics of most relevance to the study have been covered, but the level of detail for each topic varies, and I feel it may be insufficient. This is a section that I feel needs further refinement and content, that will be added as the development of this study touches on more topics.

\section{Implementation}
I have implemented a working nearest neighbour classification system, but it is quite slow. I have implemented the first few stages of a multilayer perceptron with two hidden layers, and am currently busy with the implementation of a back-propagation algorithm to optimise the network.


\section{Areas of Focus}
The next avenue of development is to process the MSTAR image chip data to facilitate quicker classification. Until now I have been using only a subset of the MSTAR data (70 image chips), but this is not feasible in the final product, as it does not include sufficient training data. The time taken to process this subset is non-negligible, and something must be done to improve the training time. The idea is to process the images in such a way that the resolution of each image, and therefore the number of pixels to process, can be reduced while retaining all the crucial data contained in each image. This will include removing the clutter surrounding each target, and may involve clustering algorithms to preserve important areas of data.

I also need to finish my implementation of the multilayer perceptron, specifically the application of the back-propagation algorithm to optimise the network.

\chapter{Image Loading and Processing}
\begin{lstlisting}[language=Python, caption=Loading and Processing Images, captionpos=b, label={list:get_images}]
def get_images(w,h,file_list=None,num_classes=8,threshold=False,noise=False):
	"""
	Load images from a file, crop/pad them to a specific size, with 
	some pre-processing options. returns an array of image vectors, 
	an array of target vectors, and a dictionary linking the 
	suffixes of each file to their respective target vector and the 
	count of each class within the dataset.

	w           --- desired width of the images
	h           --- desire height of the images
	file_list   --- if specified, list of files to read image data 
			from, otherwise uses default (default: None)
	num_classes --- number of classes into which images can be 
			classified
	threshold   --- if True, set values under the median of each 
					image to zero (default: False)
	noise       --- if True, add Gaussian noise to each image 
					(default: False)
	"""
	infile = str(num_classes) +'.txt'
	folder = 'tiffs'+str(num_classes)+'/'
	abspath = 'C:/Users/Roan Song/Desktop/thesis/'
	rng = np.random.RandomState(0)
	
	if(not file_list):
		dt = np.dtype([('filename','|S16'),
		('labels',np.int32,(num_classes,))])
		infile = 'filenames8.txt'
		filedata = np.loadtxt(infile,dtype=dt)
		file_list = [a.decode('UTF-8') 
		for a in filedata['filename']]
		file_list.sort(key=lambda x:x[-7:])
	
	suffixes = OrderedDict()
	
	for f in file_list:
		suffixes[f[-7:]] = suffixes.get(f[-7:], 0) + 1
	
	ind = 0
	for i in suffixes:    
		suffixes[i] = {"count":suffixes[i],
		"label":one_hot(ind,num_classes)}
		ind += 1    
	
	img_arr = np.zeros((len(file_list),h*w))
	target_arr = np.zeros((len(file_list),num_classes))
	i = 0
	for fname in file_list:	
		img = mpimg.imread(abspath + folder + fname)
		IN_HEIGHT = img.shape[0]
		IN_WIDTH = img.shape[1]
		
		img = pad_img(img,h,w,IN_HEIGHT,IN_WIDTH)
		image = img.reshape(h * w)
		
		if(threshold):
			below_thresh = image < np.mean(image)
			image[below_thresh] = 0
		
		image = normalise(image)
		
		if(noise):
			image += rng.normal(0,1,image.shape)
		
		img_arr[i] = image  
		target_arr[i] = suffixes[fname[-7:]]["label"]		
		i+=1 
			
	return img_arr,target_arr,suffixes
\end{lstlisting}
\newpage
\begin{lstlisting}[language=Python, caption=Generating sets, captionpos=b, label={list:gen_sets}]
def gen_sets(data,targets,train,val,test):
	"""
	Generate training, validation and test subsets from a given 
	dataset. Returns the three sets and the indices of the 
	original dataset which correspond to them
	
	data    --- full dataset to be split
	targets --- targets component of the dataset
	train   --- proportion allocated to the training set
	val     --- proportion allocated to the validation set
	test    --- proportion allocated to the test set
	
	Note: train, val and test do not have to sum to 1. 
	The unit function is applied to them, ensuring that they sum to 1.
	"""
	
	train,val,test = unit([train,val,test])
	
	training_set   = np.zeros((int(len(data)*train),2))
	validation_set = np.zeros((int(len(data)*val  ),2))
	test_set       = np.zeros((int(len(data)*test ),2))
	rng = np.random.RandomState(0)
	indices = np.arange(len(data))
	
	temp = rng.choice(indices,size=len(training_set),replace=False)
	training_indices = temp
	training_set = (np.vstack(data[temp]),np.vstack(targets[temp]))
	# training_set = (data[temp],targets[temp])
	indices = np.delete(indices,temp)
	
	temp = rng.choice(indices,size=len(validation_set),replace=False)
	validation_indices = temp
	validation_set = (np.vstack(data[temp]),np.vstack(targets[temp]))
	indices = np.delete(indices,temp)
	
	temp = rng.choice(indices,size=len(test_set),replace=False)
	testing_indices = temp
	test_set = (np.vstack(data[temp]),np.vstack(targets[temp]))
	indices = np.delete(indices,temp)
	
	return training_set, validation_set, test_set, 
	(training_indices,validation_indices,testing_indices)
\end{lstlisting}

\chapter{K-Nearest Neighbours}
\begin{lstlisting}[language=Python, caption=K Nearest Neighbours, captionpos=b, label={list:knn}]
class KNN():
	"""
	A K-Nearest Neighbours classifier
	"""
	def __init__(self,input,targets):
		""" 
		Initialisation method
		input   --- the dataset to be compared to
		targets --- the correct classes corresponding to each instance in the dataset
		"""
		self.data = input
		self.targets = targets
	
	def initD2(self,filename=None,size=None,indices=None):
		"""
		Method to initialise the squared distance array of the classifer
		This array stores the distances between every instance and every other instance
		
		filename --- a file from which the squared distance array can be imported (default: None)
		size     --- a size to which the squared distance array is to be cropped  (default: None)
		indices  --- indices of the dataset to be considered when creating the squared distance array (default: None)
		"""
		if(filename==None):
			D2 = np.zeros((len(self.data),len(self.data)))
			for i in range(len(self.data)):
				for l in range(i,len(self.data)):
					cost = 0
					if(i != l):
						for j in range(len(self.data[i])):
							cost += pow(self.data[i][j] - self.data[l][j],2)
					D2[i][l] = D2[l][i] = cost      
		
				u.progress_bar(i,len(self.data))    
		else:
			D2 = np.load(filename)
			if(indices != None):
		
				temp = np.zeros((len(indices),len(indices)))
				for y in range(len(indices)):
					for x in range(len(indices)):
						temp[y,x] = D2[indices[y],indices[x]]
		
				D2 = temp
		
			elif(size):
				D2 = D2[:size,:size]
		
		self.D2 = D2
	
	def test(self,k_arr):
		"""
		A method to test different values of K on the dataset
		returns an array of the results
		
		k_arr --- an array of K values to be tested
		"""
		results = []
		correct = np.zeros((len(k_arr))) 
		
		for img in range(len(self.data)):
			costs = sorted(list(zip(self.D2[img],self.targets.argmax(axis=1))))
			pred_lst = np.zeros((len(k_arr)))
			confidence = np.zeros((len(k_arr)))
			accuracy = np.zeros((len(k_arr)))
			ind = 0
			for k in k_arr:
				pred = list(zip(*costs[:k]))[1][1:]
				predicted_class = 0
				max = 0 
				for i in pred: 
					cnt = 0
					for l in pred:
						if(i == l):
							cnt += 1
					if(cnt > max):
						max = cnt
						predicted_class = i
		
				if(predicted_class == self.targets[img].argmax()):
					correct[ind] += 1
				confidence[ind] += max/k * 100
				accuracy[ind] = correct[ind]/len(self.data) * 100
				ind +=1 
		
		
		for x in range(ind):
			results.append([k_arr[x],correct[x],accuracy[x],confidence[x]])
		
		self.results = results  
		self.pred = list(zip(*costs[:k]))[1][1:]
		self.costs= list(zip(*costs[:k]))[0][1:]
		
		return np.array(results)
	
	def run(self,x,k,y=None):
		"""
		A method to test a single instance against the dataset
		returns the predicted class, whether or not it is correct, 
		the correct class, and a measure of confidence in the prediction
		
		x --- the input instance
		k --- the value of k determining how many neighbours to consider
		y --- the correct output if it is known (default: None)
		"""
		temp = []
		
		for img in range(len(self.data)):
			if(np.equal(self.data[img],x).all()):
				continue
			cost2 = 0
			for px in range(len(self.data[img])):
				cost2 += pow(self.data[img][px] - x[px],2)
			temp.append((cost2,self.targets[img].argmax()))
		temp = sorted(temp)
		cost = list(zip(*temp[:k]))[0]
		pred = list(zip(*temp[:k]))[1] 
		max = 0
		predicted_class = []
		for i in pred:
			cnt = 0
			for l in pred:	
				if(i == l):
				cnt += 1
			if(cnt > max):
				max = cnt
				predicted_class = i
		confidence = max/k * 100
		if(y):
		
			return  predicted_class, (y.argmax() == predicted_class), y.argmax(), confidence
		else:
			return predicted_class, confidence

\end{lstlisting}
\chapter{Multilayer Perceptron}
\begin{lstlisting}[language=Python, caption=Multilayer Perceptron, captionpos=b, label={list:mlp}]
class Multilayer_Perceptron():
def __init__(self,input,shape,num_classes,rng):
	"""
	A multilayer perceptron class
	
	input       --- a vector containing the input values
	shape       --- a tuple describing the shape of the classifier and its hidden layers
	each element in the tuple specifies the number of neurons per layer
	num_classes --- the number of classes in the dataset
	rng         --- seeded random number generator
	"""

	self.hidden_layers = []
	self.hidden_layers.append(
	HiddenLayer(input=input,n_inputs=shape[0],n_outputs=shape[1],activation=None,rng=rng))
	for i in range(2,len(shape)):
		self.hidden_layers.append(
	HiddenLayer(input=self.hidden_layers[-1].output,n_inputs=shape[i-1],n_outputs=shape[i],activation=None,rng=rng)
	
	)
	
	self.output_layer = OutputLayer(self.hidden_layers[-1].output,shape[-1],num_classes)
	self.L1 = abs(self.output_layer.weights).sum()
	self.L2 = (self.output_layer.weights**2).sum()
	self.parameters = self.output_layer.parameters
	for a in self.hidden_layers:
		self.L1 += abs(a.weights).sum()
		self.L2 += (a.weights**2).sum()
		self.parameters += a.parameters
	
	
	
	self.neg_log_likelihood = self.output_layer.neg_log_likelihood
	
	
	
	self.input = input
	self.errors = self.output_layer.errors
	self.predicted_class = self.output_layer.predicted_class
	self.weights = [a.weights for a in self.hidden_layers]
	self.weights.append(self.output_layer.weights)
	self.shape = shape
	self.rng = rng
\end{lstlisting}

\newpage
\begin{lstlisting}[language=Python, caption=Hidden Layer, captionpos=b, label={list:mlp_hidden}]
class HiddenLayer():
	"""
	This class represents a hidden layer of neurons
	It takes an array of inputs, applies an activation function to them, and returns the output
	"""
	def __init__(self,input,n_inputs,n_outputs,weights=None,bias=None,activation=T.tanh,rng=np.random.RandomState(2)):
		"""
		Initialise the hidden layer
		
		input      --- a vector containing the input values
		n_inputs   --- number of neurons feeding into the hidden layer
		n_outputs  --- number of neurons in the next layer
		weights    --- weights applied to the inputs and outputs of the hidden layer (default: None)
		bias       --- bias applied to the output values (default: None)
		activation --- activation function to be applied to neuron inputs (default: tanh)
		rng        --- seeded random number generator (default: np.random.RandomState(2))
		"""
		
		self.input = input
		if(not weights):
		weights = theano.shared(value=rng.uniform(-1,1,(n_inputs,n_outputs)),name = 'weights')
		if(not bias):
		bias = theano.shared(value=np.zeros((n_outputs,)),name='bias')
		
		self.weights = weights
		self.bias = bias
		output = T.dot(input,self.weights) + self.bias
		self.output = output if activation == None else activation(output) 
		self.parameters = [self.weights,self.bias]
\end{lstlisting}
\newpage
\begin{lstlisting}[language=Python, caption=Output Layer, captionpos=b, label={list:mlp_output}]
class OutputLayer():
	"""
	This class is a logistic regression layer for use at the output of a neural network
	"""
	def __init__(self,input,n_inputs,n_outputs):
		"""
		Initialise the output layer
		
		input     --- a vector containing the input values
		n_inputs  --- number of neurons feeding into the layer
		n_outputs --- number of classes
		"""
		
		self.weights = theano.shared(value=np.zeros((n_inputs,n_outputs)),name='weights')
		self.bias = theano.shared(value=np.zeros((n_outputs,)),name='bias')
		self.output = T.nnet.nnet.softmax(T.dot(input,self.weights)+self.bias)
		self.predicted_class = T.argmax(self.output,axis=1)
		self.parameters = [self.weights,self.bias]
		self.input = input
	
	def neg_log_likelihood(self,target):
		"""
		Returns the negative log likelihood between the classifier's output and a target
		
		target --- correct output
		"""
		return -T.mean(T.log(self.output)[T.arange(target.shape[0]),target])     
	
	def errors(self,target):
		"""
		Returns the average error between the predicted class and the target class
		
		target --- correct output
		"""
		return T.mean(T.neq(self.predicted_class,target))     
\end{lstlisting}